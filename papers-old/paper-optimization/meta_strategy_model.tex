\documentclass[english]{paper}
\usepackage[T1]{fontenc}
\usepackage[latin9]{inputenc}

\makeatletter

%%%%%%%%%%%%%%%%%%%%%%%%%%%%%% LyX specific LaTeX commands.
%% Because html converters don't know tabularnewline
\providecommand{\tabularnewline}{\\}

%%%%%%%%%%%%%%%%%%%%%%%%%%%%%% Textclass specific LaTeX commands.
\newenvironment{lyxlist}[1]
{\begin{list}{}
{\settowidth{\labelwidth}{#1}
 \setlength{\leftmargin}{\labelwidth}
 \addtolength{\leftmargin}{\labelsep}
 \renewcommand{\makelabel}[1]{##1\hfil}}}
{\end{list}}

\makeatother

\usepackage{babel}
\begin{document}

\title{High-level multi-stage optimization for Chagas disease Exploration/Exploitation
dilemma}


\author{The ``Inglorious Bug Hunters''}

\maketitle

\section{Introduction}

T. infestans detection and control are very complex. In this paper,
we apply optimization techniques to solve the high-level planning
problem. The model attempts to abstract away the targeting of specific
houses, and rather consider general strategies that we could use.


\section{Model overview}

As we discussed over the hike, we could abstractly model the planning
problem in the following way:
\begin{itemize}
\item there is a weekly budget $b$ for what we could perform.
\item time $t$ runs $1,2,$ etc weeks. we could do this for days just as
well.
\item for simplicity, we divide all the houses into two classes: ``high
yield'' and everything else.

\begin{itemize}
\item High-yield houses are those produced by reliable reports, or next
to cases with instation. These are the houses that have a relatively
good chance of being infested.
\item $H^{t}=$number of ``high-yield'' houses that we know of at time
$t$. If inspected, such houses ``yield'' infestations with probability
$y_{h}$. 
\item We start with a basic supply of ``high-yield'' houses, $H^{1}$.
Spontaneous reports are effectively like a strategy that produces
a lot of high-yield houses.
\item there is also a virtually inexhaustible supply of other houses of
which we know nothing, and which yield with probability $y_{l}\ll y_{h}$.
\item $y_{l}=0.02$, $y_{h}=0.2$
\end{itemize}
\item there are multiple ``strategies''. Abstractly all strategies have
the following parameters:

\begin{itemize}
\item $s_{i}^{t}=$ number of ``units of strategy $i$'' performed in
week $t$. There is a minumum ($0\leq s_{i}$) and a maximum ($s_{i}\leq m_{i}$).
\item $u_{i}=$number of high-yield houses inspected for T. infestans per
unit of strategy $i$. A strategy that involves going to one HY house
would have $u_{i}=1$. Many strategies are not as maximizing, and
have a mix of LY and HY.
\item $v_{i}=$number of low-yield houses inspected ... per unit of strategy
$i$. If strategy $i$ involves visiting one house, then $v_{i}+u_{v}=1$.
\item $c_{i}=$ the unit ``cost'' of strategy $i$. In general, there
could be various kinds of costs, including time, money, favors and
so on. For now, let's assume there is only one resource.
\item the gain from a strategy has, abstractly two dimensions:
\item $q_{i}(s_{i})=$expected number of ``high-yield'' houses that we
identify for future visits. In the simplest case, there is a linear
relationship: $=q_{i}s_{i}$. 
\end{itemize}
\item types of strategies

\begin{enumerate}
\item advertise on the radio
\item do active grid-based surveillance
\item visit ``high-yield'' houses
\item visit 2 hops from a source of an infestation
\end{enumerate}
\end{itemize}
Some representative values:
\begin{lyxlist}{00.00.0000}
\item [{%
\begin{tabular}{|c|c|c|c|c|c|c|c|}
\hline 
\# & Strategy & Cost$c_{i}$ & HY Visits $u_{i}$ & LY Visits $v_{i}$ & New HY ($q_{i}$) & $m_{i}$ & \tabularnewline
\hline 
\hline 
1 & Radio ad & 10 & 0 & 0 & 20 & 1 & \tabularnewline
\hline 
2 & Grid survey & 1 & 0.1 & 0.9 & 0.1 & $\infty$ & \tabularnewline
\hline 
3 & Visit known high yield & 1 & 1 & 0 & 2 & $\infty$ & \tabularnewline
\hline 
4 & Visit 2 hops from infestation & 1 & 0.5 & 0.5 & 1 & $\infty$ & \tabularnewline
\hline 
5 & Coerce houses missed in campaign & 2 & 0.9 & 0 & 0 & $\infty$ & \tabularnewline
\hline 
6 & Follow up on passive reporting & 15 & 10 & 5 & 20 & $1$ & \tabularnewline
\hline 
\end{tabular}}]~\end{lyxlist}
\begin{itemize}
\item ``Collect passive reporting'' assumes that we receive information
about 15 houses and must visit all of them. 10 of the houses are HY,
and 5 are low yield.
\item Generally, ``exploration'' strategies are those that involve a low
number of visits to HY houses.
\item Notice that some strategies might be dominated, b/c they have higher
cost and lower yields than other strategies. Therefore, they would
never be chosen in an optimal solution.
\end{itemize}

\section{Optimization Formulation}

The simplest model is the two stage (two weeks) model. This model
is not practically very useful, but it would help us to formulate
the many-stage model:
\begin{eqnarray*}
\max\sum_{t\in\{1,2\}}\sum_{i}(y_{h}u_{i}+y_{l}v_{i})s_{i}^{t}\\
\mbox{subject to all of the following:}\\
\sum_{i}c_{i}s_{i}^{1} & \leq & b\mbox{ the budget constraint for week 1}\\
\sum_{i}c_{i}s_{i}^{2} & \leq & b\mbox{ the budget constraint for week 2}\\
H^{0}-\sum_{i}u_{i}s_{i}^{1} & = & H^{1},\mbox{ \,\,}\mbox{our starting supply of HYs}\\
H^{t-1}+\sum_{i}q_{i}s_{i}^{t-1}-\sum_{i}u_{i}s_{i}^{t-1} & = & H^{t},\mbox{ \,\,}t=2\mbox{ we can only visit HYs if we have previously found them}\\
0 & \leq & H^{t}\leq\infty,\mbox{ \,\,\,\,}t=1,2\\
0 & \leq s_{i}^{t} & \leq m_{i}\mbox{ \,\,\,\,}\forall i,t
\end{eqnarray*}

\begin{itemize}
\item This problem seems computationally easy to solve, and could by done
with a very long look-ahead. It should be still doable even if we
have non-linear functions like $q_{i}(s_{i}^{1})$ instead of $q_{i}s_{i}^{1}$.
\item Notice that exploration increases the stock of high-yield houses,
which is then drained by exploration.
\item The version with the long look-ahead is an immediate generalization
of those equations.
\item The model could be dry-runned:

\begin{itemize}
\item abstractly
\item using ground-truth data, and integrated with the risk-map simulator
\end{itemize}
\end{itemize}

We can express this problem as a linear integer program \cite{wolsey1998integer} and solve using the
the R optimization package.

\section{Results}

For a simple experiment, we solve the 4-week model with a budget of
5, 25, 120. The starting number of HY sites is 3.

Budget=5. The final objective value is 3.64, which refers to the expected
number of infested sites found over 5 weeks.of searching. The strategy
map is as follows. In the map, entry $m_{it}$ for row of strategy
$i$ indicates the number of times that strategy is used on week $t$.
\begin{lyxlist}{00.00.0000}
\item [{%
\begin{tabular}{|c|c|c|c|c|c|c|c|c|}
\hline 
\# &  & Cost$c_{i}$ &  & Week 1 & Week 2 & Week 3 & Week 4 & \tabularnewline
\hline 
\hline 
1 & Radio ad & 10 &  &  &  &  &  & \tabularnewline
\hline 
2 & Grid survey & 1 &  &  &  &  &  & \tabularnewline
\hline 
3 & Visit known high yield & 1 &  & 1 & 5 & 5 & 5 & \tabularnewline
\hline 
4 & Visit 2 hops from infestation & 1 &  & 4 &  &  &  & \tabularnewline
\hline 
5 & Coerce houses missed in campaign & 2 &  &  &  &  &  & \tabularnewline
\hline 
6 & Follow up on passive reporting & 15 &  &  &  &  &  & \tabularnewline
\hline 
 & Remaining bag of HY sites & n/a &  & 0 & 1 & 6 & 11 & \tabularnewline
\hline 
\end{tabular}}]~
\end{lyxlist}
Budget=25. Objective value: 15.82
\begin{lyxlist}{00.00.0000}
\item [{%
\begin{tabular}{|c|c|c|c|c|c|c|c|c|}
\hline 
\# &  & Cost$c_{i}$ &  & Week 1 & Week 2 & Week 3 & Week 4 & \tabularnewline
\hline 
\hline 
1 & Radio ad & 10 &  & 1 &  &  &  & \tabularnewline
\hline 
2 & Grid survey & 1 &  &  &  &  &  & \tabularnewline
\hline 
3 & Visit known high yield & 1 &  &  & 25 & 25 & 25 & \tabularnewline
\hline 
4 & Visit 2 hops from infestation & 1 &  & 4 &  &  &  & \tabularnewline
\hline 
5 & Coerce houses missed in campaign & 2 &  &  &  &  &  & \tabularnewline
\hline 
6 & Follow up on passive reporting & 15 &  &  &  &  &  & \tabularnewline
\hline 
 & Remaining bag of HY sites & n/a &  & 0 & 0 & 25 & 50 & \tabularnewline
\hline 
\end{tabular}}]~
\end{lyxlist}
Budget=120. Objective value: 56.24
\begin{lyxlist}{00.00.0000}
\item [{%
\begin{tabular}{|c|c|c|c|c|c|c|c|c|}
\hline 
\# &  & Cost$c_{i}$ &  & Week 1 & Week 2 & Week 3 & Week 4 & \tabularnewline
\hline 
\hline 
1 & Radio ad & 10 &  & 2 & 2 &  &  & \tabularnewline
\hline 
2 & Grid survey & 1 &  &  & 10 &  &  & \tabularnewline
\hline 
3 & Visit known high yield & 1 &  &  &  & 100 & 100 & \tabularnewline
\hline 
4 & Visit 2 hops from infestation & 1 &  & 6 & 90 & 5 & 5 & \tabularnewline
\hline 
5 & Coerce houses missed in campaign & 2 &  &  &  &  &  & \tabularnewline
\hline 
6 & Follow up on passive reporting & 15 &  &  &  &  &  & \tabularnewline
\hline 
 & Remaining bag of HY sites & n/a &  & 0 & 0 & 18.5 & 131.0 & \tabularnewline
\hline 
\end{tabular}}]~
\end{lyxlist}

\section{Discussion}
\begin{itemize}
\item Initially (week 1 or 2), the solver indicates using strategies that
generate a lot of high-yield sites. In subsequent weeks those sites
are exploited. 
\item If the budget is large, one can use the highly-efficient strategy
of a radio ad.
\item A problem with the existing formation is that the exploitation strategy
seems to dominate other strategies in later rounds. This is because
of the assumption that it generates high returns as well as lots of
other sites that are high-yield
\item In reality, the returns from exploitation decline because the hotspot/focus
is exhausted. We need to find a way of modeling this exhaustion. This
is a spatial phenomenon. We might divide the HY sites into ``freshly-found''
vs. ``near other HY sites''. The latter generate only few new HY
sites.
\item Pitfalls: 

\begin{itemize}
\item most importantly, solving this would not give necessarily the best
houses to visit, only the right strategy mixture. This targeting problem
might be done by eye-balling, or with a separate model.
\end{itemize}
\end{itemize}

\section{Extensions}
\begin{itemize}
\item The strategy mixture would change with time. Initially, the strategies
would focus on ``exploration''/``lead generation'', and in later
stages on exploitation.
\item We could introduce a Bayesian update steps to the constants, such
as yield.
\item Even though the optimization has multiple times, after completing
one week, we recompute this with the new data. For example, we could
solve this for $t=1,2,3,\dots,T$, then do one week of work, and compute
it for $t=2,3,4,\dots T$. When we recompute, we update the variables
and parameters based on actual work performed and detection results.
For example, if we inspected a HY house and detected nothing, then
we could update the risk map, and update the $H$ based on the new
data.
\item Hotspots tend to be exhausted. To model that, we create a separate
bag of high-yield house for each hotspot.
\item We could consider $L^{t}=$number of ``low-yield'' houses, and $M^{t}$,
medium-yield that we know at time $t$. More broadly, we don't even
need to discretize the yield of the houses. rather we could say that
every strategy produces a distribution of yields, $d_{i}$ (this is
a function, or at least a vector for the expected number of houses
of each quality), per unit effort, which is added to the stock we
have access to: $D^{t+1}=D^{t}+s_{i}d_{i}$.
\item We could introduce a penalty for delaying visits to HY house. This
would prevent ``hoarding'' of targets without following them soon,
which would piss off people who report them (reduce cooperation?).
Need to think how to best write equations for this.
\end{itemize}

\section{Extension to account for space}

We need to introduce space to answer two questions:
\begin{itemize}
\item the selection of the specific houses among high yield and low yield
\item the difference in potential for new ``high yield'' uncovered 
\end{itemize}
The general idea: 

\bibliographystyle{plain}
\bibliography{inglorious}
\end{document}
